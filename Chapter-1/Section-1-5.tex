\section{随机变量的数字特征}

\subsection{数学期望}

\paragraph{随机变量的数学期望} 设随机变量 $ X $ 的分布函数为 $ F_X(x) $ ,若积分
$$ \int_{-\infty}^{+\infty} x \diff F(x) $$ 绝对收敛,
则该积分值为随机变量 $ X $ 的期望,记为 $ E(X) $ 。
\begin{itemize}[leftmargin=\paritemindent]
    \item 离散型
    $$ E(X) = \sum_{k = 1}^{\infty} x_k p(x_k) $$
    \item 连续型
    $$ E(X) = \int_{-\infty}^{\infty} x f(x) \diff x $$
\end{itemize}

\paragraph{随机变量函数的数学期望} 设随机变量 $ Y $ 时随机变量 $ X $ 的函数 $ Y = g(X) $ ,
随机变量 $ X $ 的分布函数为 $ F_X(x) $ ,则
$$ E(Y) = \int_{-\infty}^{\infty} g(x) \diff F(x) $$
\begin{itemize}[leftmargin=\paritemindent]
    \item 离散型
    $$ E(Y) = \sum_{k = 1}^{\infty} g(x_k) p(x_k) $$
    \item 连续型
    $$ E(Y) = \int_{-\infty}^{\infty} g(x) f(x) \diff x $$
\end{itemize}

\paragraph{二维随机向量函数的数学期望} 设 $ (X,Y) $ 为二维随机向量,分布函数为 $ F_{X,Y}(x,y) $ ,
又函数 $ g(x,y) $ 在 $ R^2 $ 上连续,则 $ Z = g(X,Y) $ 的数学期望为
$$ E(Z) = E[g(X,Y)] = \int_{-\infty}^{\infty} \int_{-\infty}^{\infty} g(x,y) \diff F(x,y) $$
要求广义二重积分是绝对收敛的。

\subsection{方差、协方差、相关系数}

\paragraph{方差} 设 $ X $ 是一个随机变量,若 $ E((X - E(X))^2) $ 存在,则称
$$ D(X) = E((X - E(X))^2) $$
是 $ X $ 的方差,方差的算术平方根 $ \sqrt{D(X)} = \sigma(X) $ 称为标准差。

\subparagraph{方差的含义} 方差刻划了随机变量的取值对于其数学期望的离散程度。

\subparagraph{方差的运算率}
\begin{itemize}[leftmargin=\subparitemindent]
    \item $ D(X) = E(X^2) - E^2(X) $
    \item $ D(aX + b) = a^2 D(X) $
    \item $ D(X \pm Y) = D(X) + D(Y) \pm \Cov(X,Y) $
    \item 设 $ X = (X_1, X_2, \cdots, X_n)^{\symrm{T}} $ 为 $ n $ 元随机向量, $ E(X) = \symbf{a} $ ,$ D(X) = \symbf{B} $ ,
    \begin{itemize}
        \item 对于 $ Y = \symbf{l}^{\symrm{T}}, \symbf{l} = (l_1, l_2, \cdots, l_n) $ ,
        有 $ E(Y) = \symbf{l}^{\mathrm{T}} \symbf{a}, D(Y) = \symbf{l}^{\mathrm{T}}\symbf{B}\symbf{l} $
        \item 对于 $ Y = \symbf{C}^{\symrm{T}}X, \symbf{C}_{m\times n} = (c_ij)_{m\times n} $ ,
        有 $ E(Y) = \symbf{C} \symbf{a}, D(Y) = \symbf{C}\symbf{B}\symbf{C}^{\mathrm{T}} $
    \end{itemize}
\end{itemize}

\begin{table}[htbp]
    \centering
    \caption{常见分布的期望和方差}
    \begin{tabularx}{\textwidth}{llYYY} \toprule
        \multicolumn{2}{c}{分布} & 概率质量(密度)函数 & 期望 & 方差 \\\midrule
        0-1分布 & $ 1(p) $ &  & $ p $ & $ p(1-p) $ \\\midrule
        二项分布 & $ b(n,p) $ &  & $ p $ & $ np(1-p) $ \\\midrule
        泊松分布 & $ \pi(\lambda) $ &  & $ \lambda $ & $ \lambda $ \\\midrule
        均匀分布 & $ U[a,b] $ &  & $ \ddfrac{b-a}{2} $ & $ \ddfrac{(b-a)^2}{12} $ \\\midrule
        指数分布 & $ e(\theta) $ &  & $ \theta $ & $ \theta^2 $ \\\midrule
        正态分布 & $ N(\mu, \theta^2) $ &  & $ \mu $ & $ \theta^2 $ \\\bottomrule
    \end{tabularx}
\end{table}

\paragraph{协方差} 设二维随机变量 $ (X,Y) $ 若 $ E((X - E(X))(Y - E(Y))) $ 存在,则称他为 $ X $ 与 $ y $ 的协方差,
记为 $ \Cov(X,Y) $ 。
\subparagraph{协方差的性质}

\begin{itemize}[leftmargin=\subparitemindent]
    \item $ \Cov(X,X) = D(X) $
    \item $ \Cov(X,Y) = \Cov(Y,X) $
    \item $ \Cov(a_1 X + b_1, a_2 Y + b_2) = a_1 a_2 \Cov(X, Y) $
    \item $ \Cov(X_1 + X_2, Y) = \Cov(X_1, Y) + \Cov(X_2, Y) $
    \item $ \Cov(X,Y) = E(X,Y) - E(X)E(Y) $
    \item $ \Cov(a,Y) = 0 $
\end{itemize}

\subparagraph{协方差的含义} 协方差的大小在一定程度上反映了 $ X $ 和 $ Y $ 相互间的关系,
但它还受 $ X $ 与 $ Y $ 本身度量单位的影响。

\paragraph{相关系数} 设 $ D(X) > 0, D(Y) > 0 $ 则称
\begin{equation}
    \rho_{X,Y} = \frac{\Cov(X,Y)}{\sqrt{D(X)}\sqrt{D(Y)}}
\end{equation}
为随机变量 $ X $ 和 $ Y $ 的相关系数。

\subparagraph{性质} 
\begin{itemize}[leftmargin=\subparitemindent]
    \item $ \rho_{X,Y} \leqslant 1 $
    \item $ \rho_{X,Y} = 1 $ 的充分必要条件是 $ X $ 和 $ Y $ 以概率 $ 1 $ 呈线性关系。
\end{itemize}

\subparagraph{含义} 相关系数是 $ X $ 和 $ Y $ 间线性关系的一种度量
\begin{itemize}[leftmargin=\subparitemindent]
    \item $ \left| \rho_{X,Y} \right| \rightarrow 1 $ , $ X $ 和 $ Y $ 间的线性关系越显著
    \item $ \left| \rho_{X,Y} \right| \rightarrow 0 $ , $ X $ 和 $ Y $ 间的线性关系越不显著
\end{itemize}

\subparagraph{四个等价命题}
\begin{itemize}[leftmargin=\subparitemindent]
    \item $ \rho_{X,Y} = 0 \Leftrightarrow $ $ X $ 和 $ Y $ 不相关
    \item $ E(XY) = E(X)E(Y) $
    \item $ \Cov(X,Y) = 0 $
    \item $ D(X \pm Y) = D(X) + D(Y) $
\end{itemize}

\subparagraph{不相关性与独立性的关系}
\begin{itemize}[leftmargin=\subparitemindent]
    \item 不相关: $ X $ 和 $ Y $ 之间没有线性关系,并不代表没有其它关系;
    \item 独立: $ X $ 和 $ Y $ 之前完全没有任何关系
    \item 独立 $ \Rightarrow $ 不相关,但一般情况下不相关 $ \nRightarrow $ 独立
    \item 若 $ (X,Y) $ 服从二元正太分布,则 $ X $ 和 $ Y $ 相关 $ \Leftrightarrow $  $ X $ 和 $ Y $  独立
\end{itemize}

\subsection{矩、协方差矩阵}

\paragraph{矩} 设 $ X $ 和 $ Y $ 是随机变量,有 $ k,l = 1,2,\cdots $ ,则

\subparagraph{(原点)矩} 若 $ E(X^k) $ 存在,则称它为 $ X $ 的 $ k $ 阶(原点)矩;
\subparagraph{中心矩} 若 $ E((X - E(X))^k) $ 存在,则称它为  $ X $ 的 $ k $ 阶中心矩;
\subparagraph{混合(原点)矩} 若 $ E(X^kY^l) $ 存在,则称它为 $ X $ 和 $ Y $ 的 $ k+l $ 阶混合(原点)矩;
\subparagraph{混合(原点)矩} 若 $ E((X - E(X))^k (Y - E(Y))^l) $ 存在,
则称它为 $ X $ 和 $ Y $ 的 $ k+l $ 阶混合中心矩;

\paragraph{协方差矩阵} 若 $ n $ 维随机变量的 $ n^2 $ 个二阶中心矩都存在,将他们排列成矩阵形式,
称为 $ (X_1, X_2, \cdots, X_n) $ 的协方差矩阵
\begin{equation}
    \symbf{B} = \begin{bmatrix}
        b_{11} & b_{12} & \cdots & b_{1n} \\
        b_{21} & b_{22} & \cdots & b_{2n} \\
        \vdots & \vdots & \ddots & \vdots \\
        b_{n1} & b_{n2} & \cdots & b_{nn} \\
    \end{bmatrix}
\end{equation}

\subparagraph{性质} 
\begin{itemize}[leftmargin=\subparitemindent]
    \item 对称矩阵
    \item 非负定矩阵。对于任一 $ n $ 元实列向量 $ \symbf{t} $ 有
    $$ \symbf{t}^{\symrm{T}} \symbf{Bt} \geqslant 0 $$
\end{itemize}

\subsection{两个重要不等式}

\paragraph{切比雪夫不等式} 对任意具有有限方差的随机变量 $ X $ ,都有
对 $ \forall \epsilon > 0 $ 不等式
\begin{equation}
    P\left\{ \left| X - E(X) \right| \geqslant \epsilon \right\} \geqslant \frac{D(X)}{\epsilon^2}
\end{equation}
或
\begin{equation}
    P\left\{ \left| X - E(X) \right| < \epsilon \right\} \geqslant 1 - \frac{D(X)}{\epsilon^2}
\end{equation}

\paragraph{A.L.Cauchy-Schwarz不等式} 设随机向量 $ (X,Y) $ 满足 $ E(X^2) < \infty, E(Y^2) < \infty $
则有 $$ \left[ E(XY) \right]^2 \leqslant E(X^2)E(Y^2) $$

\textbf{推论}:
$$ \Cov^2(X,Y) = [E(X - E(X))(Y - E(Y))]^2 \leqslant E((X - E(X))^2)E((Y - E(Y))^2) = D(X)D(Y) $$