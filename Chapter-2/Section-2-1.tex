\section{总体与样本}

\hparagraph{总体与个体} 研究对象的全体称为\highlight{总体},
总体中所包含的个体的个数称为总体的\highlight{容量}。
总体中每个成员称为\highlight{个体}。

\begin{itemize}[leftmargin=\paritemindent]
    \item 由于每个个体的出现是随机的,所以相应的数量指标的出现也带有随机性。
    从而可以把这种数量指标看作一个随机变量$ X $,因此随机变量$ X $的分布就是该数量指标在总体中的分布。
    \item 总体就可以用一个随机变量及其分布来描述。因此在理论上可以把总体与概率分布等同起来。
    统计中,总体这个概念的要旨是:总体就是一个随机变量(向量)或一个概率分布。
\end{itemize}

\hparagraph{样本} 
总体中抽出若干个体而成的集体,称为\highlight{样本}。样本中所含个体的个数,称为\highlight{样本容量}。

\hparagraph{抽样} 统计中,采用的抽样方法是随机抽样法,即子样中每个个体是从总体中随意地取出来的。
\hsubparagraph{抽样的分类} 
\begin{itemize}[leftmargin=\subparitemindent]
    \item \highlight{重复(返回)抽样}:从总体中抽取个体检查后放回,总体成分不变(分布不变)。
    样本 $ X_1, X_2, \cdots, X_n $ 相互独立,与总体有相同的分布。
    \item \highlight{非重复(无返回)抽样}:对有限总体取出样本后改变了总体的成分,所以 $ X_1, X_2, \cdots, X_n $ 不相互独立;
    对无限总体而言做无返回抽取,并不改变总体的成分, $ X_1, X_2, \cdots, X_n $ 相互独立,与总体有相同的分布。
\end{itemize}

\hsubparagraph{常用方法} 简单随机抽样。
\begin{itemize}[leftmargin=\subparitemindent]
    \item 代表性(随机性):。 从总体中抽取样本的每一个分量$ X_k $是随机的,每一个个体被抽到的可能性相同。
    \item 独立同分布性: $ X_1, X_2, \cdots, X_n $ 相互独立,其中每一个分量$ X_k $与所考察的总体有相同的分布。
\end{itemize}

\hsubparagraph{样本联合分布} 若总体的分布函数为 $ F(x) $ 、概率密度为 $ f(x) $ ,则其简单随机样本的联合分布函数为
\begin{equation}
    F_{X_1, X_2, \cdots, X_n}(x_1, x_2, \cdots, x_n) = F(1) F(2) \cdots F(n)
\end{equation}
其简单随机样本的联合概率密度函数为
\begin{equation}
    f_{X_1, X_2, \cdots, X_n}(x_1, x_2, \cdots, x_n) = f(1) f(2) \cdots f(n)
\end{equation}

\hparagraph{样本经验分布函数} 在 $ n $ 次独立重复实验中,事件 $ \left\{ X \leqslant x \right\}  $ 发生的频率
\begin{equation}
    \hat{F}_n(x;X_1, X_2, \cdots, X_n) = \frac{1}{n} \sum_{i=1}^{n} I(X_i \leqslant x)
\end{equation}
具有分布函数的一切性质。是在每个数据点 $ X_i $ 上权重相等的均匀分布的分布函数。

\hsubparagraph{性质}
\begin{itemize}[leftmargin=\subparitemindent]
    \item 给定 $ x $, $ \hat{F}_n(x) $ 是一个随机变量: $ n\hat{F}_n(x) $ 服从二项分布 $ b(n, F(x)) $ 
    \item $ E(\hat{F}_n(x)) = F(x) $
    \item $ D(\hat{F}_n(x)) = \frac{F(x)(1-F(x))}{n} \rightarrow 0 $
    \item $ \hat{F}_n(x) \xrightarrow{P} F(x) $
    \item Dvoretzky-Kiefer-Wolfowitz (DKW)不等式:如果 $ X_1, X_2, \cdots, X_n \sim F $ ,则对任意 $ \epsilon > 0 $
    \begin{equation}
        P\left\{ \sup_x \left| \hat{F}_n(x)) - F(x) \right| > \epsilon \right\}  \leqslant 2e^{-2n\epsilon^2}
    \end{equation}
\end{itemize}

\hsubparagraph{格列汶科定理} 当 $ n \rightarrow \infty $ 时, $ \hat{F}_n(x) $ 以概率 $ 1 $ 关于 $ x $ 一致收敛于 $ F(x) $ ,
即
\begin{equation}
    P\left\{ \lim_{n\rightarrow\infty} \sup_{-\infty<x<\infty} \left| \hat{F}_n(x) - F(x) \right| = 0 \right\} = 1
\end{equation}
当样本容量 $ n $ 足够大时,对所有的 $ x $,  $ \hat{F}_n(x) $ 与 $ F(x) $ 之差的绝对值都很小,这件事发生的概率为 $ 1 $ 。