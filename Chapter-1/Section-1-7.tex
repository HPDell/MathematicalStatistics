\section{极限定理}

\subsection{大数定律}

以 $ \mu_n $ 记实验中 $ A $ 出现的次数,
当 $ n \rightarrow \infty $ 时,频率 $ \frac{1}{n}\mu_n $ 趋近于常数 $ P(A) $。
已知
\begin{align*}
    E(\mu_n / n) & = P(A) = p \\
    D(\mu_n / n) & = \frac{pq}{n}
\end{align*}
标准化变量 $$ \xi_n = \frac{\mu_n - np}{\sqrt{npq}} $$ 则
\begin{equation}
    \lim_{n \rightarrow \infty} P\left\{ \xi_n < x \right\} = 
    \frac{1}{\sqrt{2\pi}} \int_{-\infty}^{x} e^{-\frac{t^2}{2}} \diff t
\end{equation}

\paragraph{切比雪夫大数定律} 设 $ X_1, X_2, \cdots $ 是两两不相关(相互独立)的随机变量序列,他们都有有限方差,
并且方差有共同的上界,即 $ D(X_i) \leqslant C, i = 1,2,\cdots $ ,则对 $ \forall \epsilon > 0 $ ,
\begin{equation}
    \lim_{n\rightarrow \infty} 
    P\left\{ \left| \frac{1}{n} \sum_{i=1}^{n} X_i - \frac{1}{n} E(X_i) \right| < \epsilon \right\} = 1
\end{equation}

\paragraph{独立同分布下的大数定律} 设 $ X_1, X_2, \cdots $ 是独立同分布的随机变量序列,
且 $ E(X_i) = \mu, D(X_i) = \sigma^2, i = 1,2,\cdots $ ,则对任意给 $ \epsilon > 0 $
\begin{equation}
    \lim_{n\rightarrow\infty} P\left\{ \left| \frac{1}{n} \sum_{i=1}^{n}X_i - \mu \right| < \epsilon \right\} = 1
\end{equation}

\paragraph{伯努利大数定律} 设 $ \mu_n $ 是 $ n $ 重伯努利实验中事件 $ A $ 发生的次数, $ p $ 是事件 $ A $ 发生的概率,
引入
\begin{equation*}
    X_i = \left\{\begin{array}{ll}
        1, & \text{如果第 $ i $ 次实验 $ A $ 发生} \\
        0, & \text{其他}
    \end{array}\right.
\end{equation*}
则
$$ \mu_n = \sum_{i=1}^{n}X_i $$
有
$$ \frac{\mu_n}{n} = \frac{1}{n} \sum_{i=1}^{n}X_i $$ 是事件 $ A $ 发生的频率。
设 $ p $ 是事件 $ A $ 发生的概率,则对任给的 $ \epsilon > 0 $,有
\begin{align}
    \lim_{n \rightarrow \infty} P\left\{ \left| \frac{\mu_n}{n} - p \right| < \epsilon \right\} & = 1 \\
    \lim_{n \rightarrow \infty} P\left\{ \left| \frac{\mu_n}{n} - p \right| \geqslant \epsilon \right\} & = 0 \\
\end{align}
表明,当重复实验次数$ n $充分大时,事件$ A $发生的频率$ \frac{\mu_n}{n} $与事件$ A $的概率$ p $有较大偏差的概率很小。

\subsection{中心极限定理} 

\paragraph{随机变量序列的标准化随机变量} $ n $ 个随机变量序列 $ X_1, X_2, \cdots, X_n $ 的标准化随机变量
\begin{equation}
    Z_n = \ddfrac{
        \sum_{k=1}^{n} X_k - E\left( \sum_{k=1}^{n} X_k \right)
    }{
        \sqrt{D\left(\sum_{k=1}^{n} X_k\right)}
    }
\end{equation}

\paragraph{林德贝格—莱维定理} 设 $ X_1, X_2, \cdots $ 是独立同分布的随机变量序列,
且 $ E(X_i) = \mu $ , $ D(X_i) = \sigma^2 $ , $ i = 1, 2, \cdots $ 则
\begin{equation}
    \lim_{n\rightarrow\infty} 
    P\left\{ \frac{\sum_{i=1}^{n} X_i - n\mu}{\sigma \sqrt{n}} \leqslant x \right\}
    = \int_{-\infty}^{x} \frac{1}{\sqrt{2\pi}} e^{-\frac{t^2}{2}} \diff t
\end{equation}
表明当$ n $充分大时,$ n $个具有期望和方差的独立同分布随机变量之和近似服从正态分布。

\paragraph{棣莫弗—拉普拉斯极限定理} 设 $ \mu_n $ 是 $ n $ 次伯努利实验中事件 $ A $ 出现的次数,
$ 0 < p < 1 $ ,则对任意有限区间 $ [a,b] $
\begin{itemize}[leftmargin=\paritemindent]
    \item 当 $$ a \leqslant x_k = \frac{k - np}{\sqrt{npq}} \leqslant b $$ 及 $ n \rightarrow \infty $ 时,
    一致有
    \begin{equation}
        \ddfrac{P\left\{ \mu_n = k \right\}}{\frac{1}{\sqrt{npq}}\frac{1}{\sqrt{2\pi}} e^{-\frac{1}{2}x_k^2}} \rightarrow 1
    \end{equation}
    \item 当 $ n \rightarrow \infty $ 时,一致有
    \begin{equation}
        P\left\{ a \leqslant \frac{\mu_n - np}{\sqrt{npq}} < b \right\} \rightarrow \int_{a}^{b}\phi_x \diff x
    \end{equation}
    其中 $$ \phi(x) = \frac{1}{\sqrt{2\pi}} e^{-\frac{x^2}{2}} $$
\end{itemize}