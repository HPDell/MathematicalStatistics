\section{点估计}

\subsection{点估计的方法}

\hparagraph{点估计} 我们需要构造出适当的样本的函数 $ T(X_1,X_2, \cdots, X_n) $ ,
每当有了样本,就代入该函数中算出一个值,用来作为 $ \theta $ 的估计值。
$ T(X_1,X_2, \cdots, X_n) $ 称为参数 $ \theta $ 的点估计量。
把样本值代入 $ T(X_1,X_2, \cdots, X_n) $ 中,得到 $ \theta $ 的一个车点估计值。

\hparagraph{矩估计} 用样本原点矩估计相应的总体原点矩,又用样本原点矩的连续函数估计相应的总体原点矩的连续函数,这种参数点估计法称为矩估计法。

\hsubparagraph{做法} 设总体的分布函数中含有 $ k $ 个未知参数 $ \theta_1, \theta_2, \cdots, \theta_n $
那么他的前 $ k $ 阶矩 $ \mu_1, \mu_2, \cdots, \mu_n $ 一般都是这 $ k $ 个参数的函数,记为
$$ \mu_i = \mu_i(\theta_1, \theta_2, \cdots, \theta_n), \quad i = 1, 2, \cdots, k $$
从这 $ k $ 个方程中解出
$$ \theta_i = \theta_j (\mu_1, \mu_2, \cdots, \mu_n) $$
用各 $ \mu_i $ 的估计量 $ A_i $ 分别代替上式中的各个 $ \mu_i $ 即可得到 $ \theta_i $ 的矩估计量
$$ \hat{\theta}_i = \theta_j (A_1, A_2, \cdots, A_n) $$
矩估计量的观察值称为矩估计值。

\hsubparagraph{优点} 简单易行,并不需要事先知道总体是什么分布。

\hsubparagraph{缺点} 当总体类型已知时,没有充分利用分布提供的信息。 一般场合下,矩估计量不具有唯一性。
矩估计不唯一时处理的原则是尽量使用低阶矩。

\hparagraph{最大似然法} 再一次抽样中,若得到观测值 $ x_1, x_2, \cdots, x_n $ 则选取 $ \hat{\theta}(x_1, x_2, \cdots, x_n) $
作为 $ \theta $ 的估计值,使得当 $ \theta = \hat{\theta} $ 时,样本出现的概率最大。

\hsubparagraph{做法} 设 $ x_1, x_2, \cdots, x_n $ 是 $ X_1, X_2, \cdots, X_n $ 的一个样本值, $ X $ 的联合分布率是
$$ \prod_{i=1}^{n} p(x_i;\theta), \theta \in \Theta $$
事件 $ P\left\{ X_1 = x_1, X_2 = x_2, \cdots, X_n = x_n \right\} $ 发生的概率为
\begin{align}
    \begin{split}
        \label{equ:MaxLikelyEquation}
        L(\theta) & = L(x_1, x_2, \cdots, x_n;\theta) = \prod_{i=1}^{n} p(x_i;\theta) \\
        L(\theta) & = L(x_1, x_2, \cdots, x_n;\theta) = \prod_{i=1}^{n} f(x_i;\theta) 
    \end{split}
\end{align}
式(\ref{equ:MaxLikelyEquation})称为样本似然函数。取 $ \hat{\theta} $ 使得
$$ L(x_1, x_2, \cdots, x_n;\hat{\theta}) = \max_{\theta \in \Theta} L(x_1, x_2, \cdots, x_n;\theta) $$
一般可由下式求得:
\begin{equation}
    \frac{\diff L(\theta)}{\diff \theta} = 0
\end{equation}
或
\begin{equation}
    \frac{\diff \ln L(\theta)}{\diff \theta} = 0
\end{equation}
若又多个参数,可令
\begin{align}
    \frac{\diff L(\theta)}{\diff \theta_i} & = 0 & \frac{\diff \ln L(\theta)}{\diff \theta_i} & = 0
\end{align}
求解 $ k $ 个方程组求得 $ \theta_1, \theta_2, \cdots, \theta_n $ 的最大似然估计值。

\subsection{估计量的评选标准}

\hparagraph{无偏性} 设 $ \hat{\theta}(X_1, X_2, \cdots, X_n) $ 是未知参数 $ \theta $ 的估计量,若
\begin{equation}
    E(\hat{\theta}) = \theta
\end{equation}
则称 $ \hat{\theta} $ 为 $ \theta $ 的无偏估计。

\hparagraph{有效性} 设 $ \hat{\theta}_1 = \hat{\theta}_1(X_1, X_2, \cdots, X_n),\hat{\theta}_2 = \hat{\theta}_2(X_1, X_2, \cdots, X_n) $
均是 $ \theta $ 的无偏估计,若
\begin{equation}
    D(\hat{\theta}_1) \leqslant D(\hat{\theta}_2)
\end{equation}
则称 $ \hat{\theta}_1 $ 比 $ \hat{\theta}_2 $ 有效。

\hparagraph{最小方差无偏估计} 如果存在 $ \theta $ 的无偏估计量 $ \hat{\theta}_0 $ 使得对于 $ \theta $ 的任一方差存在的无偏估计量
$ \hat{\theta} $ 都有
\begin{equation}
    D(\hat{\theta}_0) \leqslant D(\hat{\theta})
\end{equation}
则称 $ \hat{\theta}_0 $ 是 $ \hat{\theta} $ 的最小方差无偏估计量,缩写为 UMVUE 。
最小方差无偏估计是一种最优估计。基于充分完备统计量的无偏估计一定是UMVUE。

\hparagraph{优效估计} 若无偏估计 $ \hat{\theta}_0 $ 的方差 $ D(\hat{\theta}_0) = I_R $
则称 $ \hat{\theta}_0 $ 为 $ \theta $ 的优效估计。

\hsubparagraph{Cramer-Rao不等式} 设总体 $ X \sim f(x;\theta) $ , $ \theta $ 是未知参数, $ X_1, X_2, \cdots, X_n $ 是 $ X $ 的一个简单样本,
$ \hat{\theta} = \hat{\theta}(X_1, X_2, \cdots, X_n) $ 是 $ \theta $ 的无偏估计,则
\begin{equation}
    D(\hat{\theta}) \geqslant \frac{1}{nI(\theta)}
\end{equation}
右端称为C-R下界,记为$ I_R $。有时能找到无偏估计使它的方差达到这个下界,有时达不到。

\hsubparagraph{Fisher信息量} 设总体 $ X $ 的概率密度函数为 $ f(x;\theta), \theta \in \Theta $ 且满足下列条件(称为正则条件),
则称\begin{equation}
    I(\theta) = E\left(\frac{\partial \ln f(X;\theta)}{\partial \theta}\right)^2
\end{equation}
为总体分布的 Fisher 信息量。
\begin{itemize}[leftmargin=\subparitemindent]
    \item $ \Theta $ 是实数轴上的一个开区间;
    \item 支撑 $ S = \left\{ x \left| f(x;\theta) > 0 \right. \right\} $ 与 $ \theta $ 无关;
    \item $ \frac{\partial f(x;\theta)}{\partial \theta} $ 存在且对 $ \Theta $ 中一切 $ \theta $ 有 \begin{equation}
        \frac{\partial}{\partial\theta} \int_{-\infty}^{\infty} f(x;\theta) \diff x
        = \int_{-\infty}^{\infty} \frac{\partial}{\partial\theta} f(x;\theta) \diff x
    \end{equation}
    \item $ E\left(\frac{\partial \ln f(X;\theta)}{\partial \theta}\right)^2 $ 存在。
\end{itemize}

\hsubparagraph{Fisher信息矩阵} 若 $ \theta $ 为向量,则称
\begin{equation}
    I(\theta) = E \left[ 
        \left( \frac{\partial f(X|\theta)}{\partial\theta} \right)
        \left( \frac{\partial f(X|\theta)}{\partial\theta} \right)^{\symrm{T}}
    \right]
\end{equation}
为总体分布的Fisher信息矩阵。

\hsubparagraph{Fisher信息量的另一种表达式} 若 $ \theta $ 为标量,
$ \frac{\partial^2 \ln f(X;\theta)}{\partial \theta^2} $ 存在且满足正则条件,则
\begin{equation}
    I(\theta) = - E\left(\frac{\partial^2 \ln f(X;\theta)}{\partial \theta^2}\right)
\end{equation}
若 $ \theta $ 为向量,
$ \frac{\partial^2 \ln f(X;\theta)}{\partial \theta \partial \theta^{\symrm{T}}} $ 存在且满足正则条件,则
\begin{equation}
    I(\theta) = - E \left( \frac{\partial^2 \ln f(X;\theta)}{\partial \theta \partial \theta^{\symrm{T}}} \right)
\end{equation}

\hparagraph{相合性(一致性)} 设 $ \hat{\theta}(X_1, X_2, \cdots, X_n) $ 是参数 $ \theta $ 的估计量,
若对于任意 $ \theta \in \Theta $ 当 $ n \rightarrow \infty $ 时 $ \hat{\theta}(X_1, X_2, \cdots, X_n) $ 依概率收敛到 $ \theta $,
则称 $ \hat{\theta} $ 为 $ \theta $ 的相合估计量。即对于任意 $ \epsilon > 0 $ 有
\begin{equation}
    \lim_{n\rightarrow\infty} P\left\{ \left| \hat{\theta} - \theta \right| < \epsilon \right\}  = 1, \theta \in \Theta
\end{equation}
