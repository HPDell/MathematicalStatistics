\section{单因素实验的方差分析}

\hparagraph{数学模型} 设在实验中,因素 $ A $ 有 $ S $ 个不同水平 $$ A_1, A_2, \cdots, A_s $$
在水平下的实验结果 $ X_j \sim N(\mu_j, \sigma^2) (j = 1, 2, \cdots, s) $ 。
其中 $ \mu_j $ 和 $ \sigma^2 $ 是位置参数。在水平 $ A_j $ 下做 $ n_j $ 次独立实验,得到结果
$$ \begin{array}{|l|llll|}\hline
    \mbox{序号} & A_1 & A_2 & \cdots & A_s \\\hline
    1           & X_{11} & X_{12} & \cdots & X_{1s} \\
    2           & X_{21} & X_{22} & \cdots & X_{2s} \\
    3           & X_{31} & X_{32} & \cdots & X_{3s} \\
    \vdots      & \vdots & \vdots & \ddots & \vdots \\
    n_j         & X_{n_11} & X_{n_22} & \cdots & X_{n_ss} \\\hline
    \mbox{均值} & \bar{X}_{•1} & \bar{X}_{•2} & \cdots & \bar{X}_{•s} \\\hline
\end{array} $$
$ X_{1j}, X_{2j}, \cdots, X_{n_jj} $ 是来自总体 $ X_j $ 的容量为 $ n_j $ 的一个样本,其观察值为
$ x_{1j}, x_{2j}, \cdots, x_{n_jj} $ 。检验假设
\begin{align*}
    H_0 &: \mu_1 = \mu_2 = \cdots = \mu_s \\
    H_1 &: \mu_1, \mu_2, \cdots, \mu_n \mbox{不全相等}
\end{align*}
由于 $ X_{ij} $ 相互独立,且 $ X_{ij} \sim N(\mu_j,\sigma^2)(i=1, 2, \cdots, n_j; j = 1, 2, \cdots, s) $ ,
若记 $ \epsilon_{ij} = X_{ij} - \mu_j $ 则 $ \epsilon_{ij} \sim N(0, \sigma^2) $ 且互相独立。
则有
\begin{equation}
    \left\{\begin{array}{l}
        X_{ij} = \mu_j + \epsilon_{ij} \\
        \epsilon_{ij} \sim N(0, \sigma^2)
    \end{array}\right.
\end{equation}
其中,$ i = 1, 2, \cdots, n_j $ , $ j = 1, 2, \cdots, s $ , $ \mu_j $ 与 $ \sigma^2 $ 均为未知参数。

\hparagraph{线性可加模型} 令
\begin{equation*}
    \left\{\begin{array}{>{\displaystyle\arraybackslash}l}
        \mu = \frac{1}{n} \sum_{j=1}^{s} n_j \mu_j, \quad i = 1, 2, \cdots, m \\
        \delta_j = \mu_j - \mu , \quad n = \sum_{j=1}^{s} n_j
    \end{array}\right.
\end{equation*}
则 $ \mu $ 是各水平下总体均值的加权平均,称为总平均值;
$ \delta_i $ 代表了第 $ j $ 水平下的总体均值和平均值的差异,称为 $ A_j $ 的效应,满足
$$ \sum_{j=1}^{s} n_j \delta_j = 0 $$
得到
\begin{equation}
    \left\{\begin{array}{D}
        X_{ij} = \mu + \delta_j + \epsilon_{ij} \\
        \sum_{i=1}^{m} n_j \delta_j = 0
    \end{array}\right.
\end{equation}
其中, $ i = 1, 2, \cdots, n_j $ , $ j = 1, 2, \cdots, s $ , $ \epsilon_{ij} \sim N(0, \sigma^2) $ 且相互独立。
该模型的等价假设为
\begin{align*}
    H_0 &: \delta_1, \delta_2, \cdots, \delta_s = 0 \\
    H_1 &: \exists j, \delta_j \neq 0
\end{align*}

\hparagraph{总平方和的分解} 总离差平方和为
\begin{align}
    \begin{split}
        S_T 
        & = \sum_{j=1}^{s} \sum_{i=1}^{n_j} (X_{ij} - \bar{X})^2  \\
        & = \sum_{j=1}^{s}\sum_{i=1}^{n_j} \left[(X_{ij} - \bar{X}_{•j}) - (\bar{X}_{•j} - \bar{X})\right]^2 \\
        & = \sum_{j=1}^{s}\sum_{i=1}^{n_j} \left[(X_{ij} - \bar{X}_{•j})\right]^2 
        - \sum_{j=1}^{s}\sum_{i=1}^{n_j} \left[(\bar{X}_{•j} - \bar{X})\right]^2 \\
        & = S_E + S_A
    \end{split}
\end{align}

\hsubparagraph{组内离差和组间离差}
\begin{itemize}[leftmargin=\subparitemindent]
    \item $ S_E $ 组内离差,\begin{equation}
        S_E = \sum_{j=1}^{s}\sum_{i=1}^{n_j} \left[(X_{ij} - \bar{X}_{•j})\right]^2 
    \end{equation}反映因素 $ A $ 各水平下的子样均值和样本值之间的差异,是由随机误差引起的,也叫误差平方和。
    \item $ S_A $ 组间离差,\begin{equation}
        S_A = \sum_{j=1}^{s}\sum_{i=1}^{n_j} \left[(\bar{X}_{•j} - \bar{X})\right]^2
        = \sum_{j=1}^{s}n_j \left(\bar{X}_{•j} - \bar{X}\right)^2
    \end{equation}反映因素 $ A $ 各水平下的子样均值和总平均值之间的差异,也叫因素 $ A $ 效应的平方和。
\end{itemize}

\hsubparagraph{$ S_T,S_E,S_A $ 的统计特性} \begin{align}
    \frac{S_T}{\sigma^2} & \sim \chi^2(n-1) \\
    \frac{S_E}{\sigma^2} & \sim \chi^2(n-s) \\
    \frac{S_A}{\sigma^2} & \sim \chi^2(s-1) 
\end{align}

\hparagraph{检验方法} 在 $ H_0: \delta_i = 0 $ 成立的条件下,统计量
\begin{equation}
    F = \ddfrac{\hfrac{\frac{S_A}{\sigma^2}}{(s-1)}}{\hfrac{\frac{S_E}{\sigma^2}}{(n-s)}}
\end{equation}
有 $ F \sim F(s-1, n-s) $ 。给定 $ \alpha $ 则:

\begin{itemize}[leftmargin=\paritemindent]
    \item 若 $ F > F_\alpha(s-1, n-s) $ ,则拒绝 $ H_0 $ ;
    \item 若 $ F < F_\alpha(s-1, n-s) $ ,则接受 $ H_0 $ 。
\end{itemize}

\hparagraph{一元方差分析表} 如 表\ref{tab:一元方差分析表} 所示

\begin{table}[hbtp]
    \renewcommand{\arraystretch}{2}
    \centering
    \caption{一元方差分析表}
    \label{tab:一元方差分析表}
    \begin{tabular}{l|l|l|l|l} \hline
        方差来源 & 平方和 & 自由度 & 均方 & $ F $值 \\\hline
        因素 $ A $ & $ S_A $ & $ s-1 $ & $ \displaystyle \bar{S}_A = \frac{S_A}{s-1} $ & $ \displaystyle F = \frac{\bar{S}_A}{\bar{S}_E} $ \\
        误差 $ E $ & $ S_E $ & $ n-s $ & $ \displaystyle \bar{S}_E = \frac{S_E}{n-s} $ & \\
        总和 $ T $ & $ S_T $ & $ n-1 $ & & \\\hline
    \end{tabular}
\end{table}

\hparagraph{未知参数的估计} 未知参数 $ \mu, \mu_j, \sigma^2 $ 
\begin{align}
    \hat{\sigma}^2 & = \frac{S_E}{n-s} \\
    \hat{\mu} & = \bar{X} \\ 
    \hat{\mu}_j & = \bar{X}_{•j}
\end{align}