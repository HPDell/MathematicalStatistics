\section{一元回归}

\subsection{一元线性回归}

\hparagraph{一元线性回归模型} \begin{equation}
    y = \beta_0 + \beta_1 x + \epsilon
\end{equation}

\hparagraph{一元线性回归方程} \begin{equation}
    \hat{y} = \hat{\beta}_0 + \hat{\beta}_1 x
\end{equation}

\hparagraph{最小二乘估计} 设 $ (x_1, y_1), (x_2, y_2), \cdots, (x_n, y_n) $ 是 $ (x,y) $ 的一组观测值,
对每个样本观测值 $ (x_i, y_i) $ 考虑 $ y_i $ 与其回归值
\begin{eqnarray}
    E(y_i) = \beta_0 + \beta_1 x_i
\end{eqnarray}
的离差
\begin{equation}
    y_i - E(y_i) = y_i - \beta_0 - \beta_1 x_i
\end{equation}
定义离差平方和
\begin{equation}
    Q(\beta_0, \beta_1) = \sum_{i=1}^{n} (y_i - E(y_i))^2 = \sum_{i=1}^{n}(y_i - \beta_0 - \beta_1 x)^2
\end{equation}
使其最小化。由
\begin{align*}
    \left. \frac{\partial Q}{\partial\beta_0} \right|_{(\hat{\beta}_0, \hat{\beta}_1)} & = 0 \\
    \left. \frac{\partial Q}{\partial\beta_1} \right|_{(\hat{\beta}_0, \hat{\beta}_1)} & = 0
\end{align*}
得正规方程组
\begin{equation}
    \left\{\begin{array}{l}
        n\hat{\beta}_0 + n\bar{x} \hat{\beta}_1 = n\bar{y} \\
        n\bar{x}\hat{\beta_0} + \left( \displaystyle{\sum_{i=1}^{n}x_i^2} \right) \hat{\beta}_1 = \sum_{i=1}^{n}x_iy_i
    \end{array}\right.
\end{equation}
解得
\begin{align}
    \hat{\beta}_1 & = \ddfrac{\sum_{i=1}^{n}(x_i - \bar{x})(y_i - \bar{y})}{\sum_{i=1}^{n}(x_i - \bar{x})^2} \\
    \hat{\beta}_0 & = \bar{y} - \hat{\beta}_1 \bar{x}
\end{align}

\hsubparagraph{简记} \begin{align}
    L_{xx} &= \sum_{i=1}^{n} (x_i - \bar{x})^2 = \sum_{i=1}^{n}x_i^2 - n \bar{x}^2 \\
    L_{xy} &= \sum_{i=1}^{n} (x_i - \bar{x})(y_i - \bar{y}) = \sum_{i=1}^{n}x_iy_i - n\bar{x}\bar{y} \\
    L_{yy} &= \sum_{i=1}^{n} (y_i - \bar{y})^2 = \sum_{i=1}^{n}y_i^2 - n \bar{y}^2
\end{align}
则 \begin{align}
    \hat{\beta}_1 &= \frac{L_{xy}}{L_{xx}} \\
    \hat{\beta}_0 &= \bar{y} - \hat{\beta}_1 \bar{x}
\end{align}

\hsubparagraph{矩阵表示} 记 \begin{align}
    y & = \begin{pmatrix}
        y_1 \\ y_2 \\ \vdots \\ y_n
    \end{pmatrix} &
    X & = \begin{pmatrix}
        1 & x_1 \\ 1 & x_2 \\ \vdots & \vdots \\ 1 & x_n
    \end{pmatrix} &
    A & = \begin{pmatrix}
        \hat{\beta}_0 \\ \hat{\beta}_1
    \end{pmatrix}
\end{align}
则一元回归方程的最小二乘解为
\begin{equation}
    A = (X^{\symrm{T}}X)^{-1} X^{\symrm{T}}y
\end{equation}

\hsubparagraph{最小二乘估计量的性质}
\begin{itemize}[leftmargin=\subparitemindent]
    \item $ \hat{\beta}_0, \hat{\beta}_1 $ 都是 $ y_1, y_2, \cdots, y_n $ 的线性组合;
    \item $ \hat{\beta}_0, \hat{\beta}_1 $ 的最小二乘估计都是无偏的;
    \item 有 \begin{align}
        \hat{\beta}_0 & \sim N \left( \beta_0 , \left(\frac{1}{n} + \frac{\bar{x}}{L_{xx}}\right)\sigma^2 \right) \\
        \hat{\beta}_1 & \sim N \left( \beta_1 , \frac{\sigma^2}{L_{xx}} \right) \\
        \Cov(\hat{\beta}_0, \hat{\beta}_1) &= - \frac{\bar{x}}{L_{xx}} \sigma^2
    \end{align}
    且 $ \hat{\sigma}^2 $ 与 $ \left( \hat{\beta}_0, \hat{\beta}_1 \right) $ 相互独立。
\end{itemize}


\hparagraph{$ \sigma^2 $的估计} $ \sigma^2 $ 的无偏估计为 \begin{equation}
    \hat{\sigma}^2  = \frac{Q_e}{n-2}
\end{equation}
其中 $ Q_e $ 使残差平方和,即因随机因素引起的误差
\begin{equation}
    Q_e = \sum_{i=1}^{n} e_i^2 = \sum_{i=1}^{n}(y_i - \hat{y}_i)^2 = L_{yy} - \hat{\beta}_1 L_{xy}
\end{equation}

\hparagraph{线性回归方程的显著性检验}
\hsubparagraph{$ F $ 检验}  总离差平方和、回归平方和、残差平方和满足 $ S_{\mbox{\small 总}}^{2} = S_{\mbox{\small 回}}^{2} + S_{\mbox{\small 残}}^{2} $ ,
其中
\begin{align}
    S_{\mbox{\small 总}}^{2} & = \sum_{i=1}^{n} (y_i - \bar{y})^2 \\
    S_{\mbox{\small 回}}^{2} & = \sum_{i=1}^{n} (\hat{y}_i - \bar{y})^2 \\
    S_{\mbox{\small 残}}^{2} & = \sum_{i=1}^{n} (y_i - \hat{y})^2
\end{align}

\begin{itemize}[leftmargin=\subparitemindent]
    \item $ S_{\mbox{\small 回}}^{2} $ 反映了由于自变量 $x$ 的变化引起的因变量 $y$ 的差异,体现了 $x$ 对 $y$ 的影响;
    \item $ S_{\mbox{\small 残}}^{2} $ 反映了种种其它因素对 $y$ 的影响,这些因素没有反映在自变量中,它们可作为随机因素看待。
    \item 当 $ \beta_1 = 0 $ 时, $ S_{\mbox{\small 回}}^{2} $ 与 $ S_{\mbox{\small 残}}^{2} $ 相互独立,且
    \begin{align*}
        \frac{S_{\mbox{\small 回}}^{2}}{\sigma^2} & 
        = \frac{1}{\sigma^2} \sum_{i=1}^{n}(\hat{y}-\bar{y})^2 
        = \hat{\beta}_1^2 L_{xx} = \hat{\beta}_1 L_{xy} \\
        & \sim \chi^2(1) \\
        \frac{S_{\mbox{\small 残}}^{2}}{\sigma^2} & 
        = \frac{1}{\sigma^2} \sum_{i=1}^{n}(y-\hat{y})^2 
        = L_{yy} - \hat{\beta}_1 L_{xy} \\
        & \sim \chi^2(n-2)
    \end{align*}
\end{itemize}

\subparagraph*{} {\hsubparagraphcontinue 检验假设 $ H_0 : \beta_1 = 0, H_1 : \beta_1 \neq 0 $ ,统计量
\begin{equation}
    F = \frac{(n-1)S_{\mbox{\small 回}}^{2}}{S_{\mbox{\small 残}}^{2}} \sim F(1, n-2)
\end{equation}
对于给定的显著性水平 $ \alpha(0 < \alpha < 1) $ , $ H_0  $ 的拒绝域为
\begin{equation}
    F > F_\alpha(1,n-2)
\end{equation}}

\hsubparagraph{$ t $检验} 检验假设 $ H_0 : \beta_1 = 0, H_1 : \beta_1 \neq 0 $ 。
当原假设成立时,有 $ \hat{\beta}_1 \sim N\left(0, \frac{\sigma^2}{L_{xx}}\right) $ 即
$$ \frac{\hat{\beta}_1\sqrt{L_{xx}}}{\sigma} \sim N(0,1) $$
又有
$$ \frac{Q_e}{\sigma^2} \sim \chi^2(n-2) $$ 则有
\begin{equation}
    t = \ddfrac{
        \frac{\hat{\beta}_1\sqrt{L_{xx}}}{\sigma}
    }{
        \sqrt{\ddfrac{\frac{Q_e}{\sigma^2}}{(n-2)}}
    } = \frac{\hat{\beta}_1\sqrt{L_{xx}}}{\hat{\sigma}} \sim t(n-2)
\end{equation}
拒绝域为 $ \left| t \right| \geqslant t_{\frac{\alpha}{2}}(n-2) $ 。

\hsubparagraph{$ r $ 检验} 相关系数
\begin{equation}
    r^2 = \frac{L_{xy}^2}{L_{xx}^2 L_{yy}^2} = \frac{S_{\mbox{\small 回}}}{S_{\mbox{\small 总}}}
\end{equation}
$ r^2 $ 越大,说明 $ x $ 的变化引起 $ y $ 的变化就越大。

\begin{itemize}[leftmargin=\subparitemindent]
    \item $ \left| r \right| \leqslant 1 $ ;
    \item $ \left| r \right| \neq 0 $ 时,$ y $ 与 $ x $ 具有线性相关关系;
    \item $ \left| r \right| $ 越大,变量 $ y $ 与 $ x $ 之前的线性相关程度越强。
\end{itemize}

\hparagraph{回归系数的区间估计} 

\hsubparagraph{$ \hat{\beta}_1 $} 置信水平为 $ 1- \alpha $ 的置信区间为
\begin{equation}
    \left( \hat{\beta}_1 - \frac{\hat{\sigma}}{\sqrt{L_{xx}}} t_{\frac{\alpha}{2}}(n-2) ,
    \hat{\beta}_1 + \frac{\hat{\sigma}}{\sqrt{L_{xx}}} t_{\frac{\alpha}{2}}(n-2) \right)
\end{equation}

\hsubparagraph{$ \hat{\beta}_0 $} 置信水平为 $ 1- \alpha $ 的置信区间为
\begin{equation}
    \left( \hat{\beta}_1 - \hat{\sigma} \sqrt{\frac{1}{n} + \frac{\bar{x}^2}{L_{xx}}} t_{\frac{\alpha}{2}}(n-2) ,
    \hat{\beta}_1 + \hat{\sigma} \sqrt{\frac{1}{n} + \frac{\bar{x}^2}{L_{xx}}} t_{\frac{\alpha}{2}}(n-2) \right)
\end{equation}

\hparagraph{预测}

\hsubparagraph{单值预测} 对于任意给定的 $ x = x_0 $ , $ y_0 $ 的预测值为 $ \hat{y}_0 = \hat{\beta}_0 + \hat{\beta}_1 x_0 $ 。

\hsubparagraph{区间预测} 置信水平为 $ 1- \alpha $ 的置信区间为
\begin{equation}
    \left(
        \hat{y}_0 - \hat{\sigma}\sqrt{1 + \frac{1}{n} + \frac{(x_0 - \bar{x})^2}{L_{xx}}} t_{\frac{\alpha}{2}}(n-2),
        \hat{y}_0 + \hat{\sigma}\sqrt{1 + \frac{1}{n} + \frac{(x_0 - \bar{x})^2}{L_{xx}}} t_{\frac{\alpha}{2}}(n-2)
    \right)
\end{equation}
当 $ n $ 较大时, $ t_{\frac{\alpha}{2}}(n-2) \approx u_{\frac{\alpha}{2}} $ ,
$ 1 + \frac{1}{n} + \frac{(x_0 - \bar{x})^2}{L_{xx}} \approx 1 $ ,
则置信水平为 $ 1- \alpha $ 的置信区间为 
\begin{equation}
    \left(
        \hat{y}_0 - \hat{\sigma}u_{\frac{\alpha}{2}},
        \hat{y}_0 + \hat{\sigma}u_{\frac{\alpha}{2}}
    \right)
\end{equation}

\hparagraph{控制} 要求 $ y = \beta_0 + \beta_1 x + \epsilon $ 的值以 $ 1 - \alpha $ 的概率落在指定区间 $ (y', y'') $ 。
控制 $ x $ 满足以下两个不等式
\begin{align}
    \hat{y} - \delta(x') & \geqslant y' \\
    \hat{y} + \delta(x'') & \leqslant y''
\end{align}
成立,则区间 $ (x', x'') $ 就是所求 $ x $ 的控制区间。

\subsection{非线性回归}

\LTXtable{\textwidth}{Chapter-5/Table-5-1.tex}