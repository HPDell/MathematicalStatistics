\section{随机变(向)量及其分布}

\subsection{随机变量及其分布}

\paragraph{随机变量} 设随机试验的样本空间  $ \Omega = {\omega} $, $ \xi = \xi(\omega) $
是定义在样本空间 $ \Omega $ 上的\textbf{实值单值函数},称 $ \xi = \xi(\omega) $ 是随机变量。

\subparagraph{随机变量和普通函数的区别}
\begin{itemize}[leftmargin=\subparitemindent]
    \item 定义域不同:随机变量定义在样本空间 $ \Omega $ 上,定义域可以是数也可以不是数;
    而普通函数是定义在实数域上。
    \item 随机变量函数的取值在试验之前无法确定,且取值有一定的概率;而普通函数却没有。
\end{itemize}

\subparagraph{随机变量的分类}
\begin{itemize}[leftmargin=\subparitemindent]
    \item 离散型随机变量
    \item 连续型随机变量
    \item 混合型随机变量
\end{itemize}

\paragraph{分布函数} 设 $ X $ 是一个随机变量, $ x $ 是任意实数,称函数 \begin{equation}
    \label{equ:分布函数定义}
    F(x) = P\{X < x\} \quad (-\infty < x < +\infty)
\end{equation}
为 $ X $ 的分布函数。分布函数 $ F(x) $ 的值就表示 $ X $ 落在区间 $ (-\infty,x] $ 上的概率。

\subparagraph{特点} 
\begin{itemize}[leftmargin=\subparitemindent]
    \item 分布函数完整描述了随机变量的统计规律性
    \item 分布函数是一个普通实值函数 
\end{itemize}

\subparagraph{性质} 以下三条性质是判断函数是否是分布函数的充要条件。
\begin{itemize}[leftmargin=\subparitemindent]
    \item 单调不减
    \item  $ 0 \leqslant F(x) \leqslant 1 $ 且 \begin{align}
        \begin{split}
            F(-\infty) &= \lim_{x \rightarrow -\infty} F(x) = 0 \\
            F(+\infty) &= \lim_{x \rightarrow +\infty} F(x) = 1
        \end{split}
    \end{align}
    \item 右(左)连续性:\begin{align}
        \begin{split}
            F(x - 0) &= \lim_{x \rightarrow x - 0} F(y) = F(x) \\
            F(x + 0) &= \lim_{x \rightarrow x + 0} F(y) = F(x)
        \end{split}
    \end{align}
\end{itemize}

\subparagraph{常用的概率公式}
\begin{itemize}[leftmargin=\subparitemindent]
    \item  $ P(a < X \leqslant b) = P(X \leqslant b) - P(X \leqslant a) = F(b) - F(a) $ 
    \item  $ P(X = x_0) = P(X \leqslant x_0) - P(X < x_0) = F(x_0) - F(x_0 - 0) $ 
    \item  $ P(X \geqslant x_0) = 1 - P(X < x_0) = 1 - F(x_0 - 0) $ 
    \item  $ P(X > x_0) = 1 - P(X \leqslant x_0) = 1 - F(x_0) $ 
\end{itemize}

\subsubsection{离散型随机变量及其分布}

\paragraph{离散型随机变量} 若随机变量 $ X $ 的全部可能取值是有限个或可列无限多个,
则称此随机变量是离散型随机变量。

\paragraph{分布律} 设离散型随机变量 $ X $ 的所有可能取值为 $ x_k,k=(1,2,\cdots) $ ,
 $ X $ 取各个可能值的概率为 $ P(X = x_k) = p_k $ , $ p_k $ 满足
\begin{itemize}[leftmargin=\paritemindent]
    \item  $ p_k \geqslant 0 $ 
    \item  $ \sum_{k=1}^\infty p_k = 1 $ 
\end{itemize}
则称 $ p_k $ 为离散型随机变量X的概率分布或分布律。

\paragraph{常用的离散型随机变量分布}

\subparagraph{0—1分布}  $ P(X = k) = p^k(1-p)^{1-k}, k = 0,1 $ 

\subparagraph{二项分布}  $ P(X=k) = C_n^k p^k (1-p)^{n-k}, k = 0,1,\cdots,n. $ 记为 $ X \sim b(n,p) $
\begin{itemize}[leftmargin=\subparitemindent]
    \item \textbf{含义}:   $ n $  重贝努里试验中出现“成功”次数 $ X $ 的概率分布。
    \item \textbf{退化为0—1分布}:  $ n=1 $ 
\end{itemize}

\subparagraph{泊松分布}  分布律  $$ P(X=k) = \frac{\lambda^k}{k!}e^{-\lambda} \quad k = 0,1,2,\cdots $$ 
其中  $ \lambda $ 是常数。记为  $ X \sim P(\lambda) $ 。
泊松分布在管理科学、运筹学以及自然科学的某些问题中都占有重要的地位。 
\begin{itemize}[leftmargin=\subparitemindent]
    \item 排队问题:在一段时间内窗口等待服务的顾客人数
    \item 生物存活的个数
    \item 放射的粒子数
\end{itemize}

\subsubsection{连续型随机变量及其分布}

\paragraph{连续性随机变量} 如果随机变量 $ X $ 的分布函数为
$$ F(x) = \int_{-\infty}^x f(t)\diff t $$ 
其中被积函数 $ f(t) \geqslant 0 $ 则 $ X $ 为连续型随机变量,
称 $ f(t) $ 为概率密度函数或概率密度。

\paragraph{概率密度的性质}
\begin{itemize}[leftmargin=\paritemindent]
    \item  $ f(x) \geqslant 0 $ 
    \item  $ \int_{-\infty}^\infty f(x) \diff x = 1 $ 
    \item  $ P(a < X \leqslant b) = F(b) - F(a) = \int_a^b f(x) \diff x $
    \item  $ P(X = a) = 0 $ 
    \item 在 $ f(x) $ 的连续点 $ x $ 处,有\begin{align}
        \begin{split}
            f(x) = F'(x)
            & = \lim_{\Delta x \rightarrow 0^+} \frac{F(x + \Delta x) - F(x)}{\Delta x} \\
            & = \lim_{\Delta x \rightarrow 0^+} \frac{P(x < X \leqslant (x + \Delta x))}{\Delta x} \\
        \end{split}
    \end{align}
    \item  $ P(a \leqslant X \leqslant b) = P(a < X \leqslant b) = P(a \leqslant X < b) = P(a < X < b) $ 
    \item 若已知连续型随机变量 $ X $ 的密度函数为 $ f(x) $ ,则 $ X $ 在任意区间 $ G $ 上取值的概率为
    \begin{equation}
        P\{X \in G\} = \int_G f(x) \diff x
    \end{equation}
\end{itemize}

\paragraph{常见的连续型随机变量的分布}

\subparagraph{均匀分布}  $ X $ 的概率密度为:
\begin{equation}
    f(x) = \left\{ \begin{array}{ll}
        \ddfrac{1}{b-a}, & x \in (a,b) \\
        0, & x \in (-\infty, a] \cup [b, \infty)
    \end{array} \right.
\end{equation}
则称 $ X $ 服从 $ (a,b) $ 上的均匀分布,记为 $ X \sim U(a,b) $ 。分布函数为
\begin{equation}
    F(x) = \int_{-\infty}^x f(t) \diff t = \left\{ \begin{array}{ll}
        0, & x < a \\
        \ddfrac{x-a}{b-a}, & a \leqslant x < b  \\
        1, & x \geqslant b
    \end{array} \right.
\end{equation}

\subparagraph{指数分布} $ X $ 的概率密度为:
\begin{equation}
    f(x) = \left\{ \begin{array}{ll}
        \lambda e^{-\lambda x} & x \geqslant 0 \\
        0 & x < 0
    \end{array} \right.
\end{equation}
则称 $ X $ 服从参数为 $ \lambda $ 的指数分布,记为 $ X \sim E(\lambda) $ 。分布函数为
\begin{equation}
    F(x) = \left\{ \begin{array}{ll}
        1 - \lambda e^{-\lambda x} & x \geqslant 0 \\
        0 & x < 0
    \end{array} \right.
\end{equation}

\subparagraph{正态分布} $ X $ 的概率密度为:
\begin{equation}
    f(x) = \frac{1}{\sqrt{2\pi} \sigma} e ^{- \frac{(x-\mu)^2}{2 \sigma^2}} \quad -\infty < x < \infty
\end{equation}
其中 $ \mu $ 和 $ \sigma $ 时常数,且 $ \sigma > 0 $ ,
则称 $ X $ 服从参数为 $ \mu, \sigma $ 的正态分布,或高斯分布,记为 $ X \sim N(\mu, \sigma^2) $ 。

\paragraph{正态分布}
\subparagraph{图形特点} 正态分布的密度曲线时一条关于 $ \mu $ 对称的钟形曲线,特点是:
“两头小,中间大,左右对称”
\begin{itemize}[leftmargin=\subparitemindent]
    \item $ \mu $ 决定了图形的中心位置
    \item $ \sigma $ 决定了图形中峰的陡峭程度
\end{itemize}

\subparagraph{标准正态分布} $ \mu = 0, \sigma = 1 $ 的分布被称为标准正态分布,
其密度函数和分布函数常用 $ \phi(x) $ 和 $ \Phi(x) $ 表示
$$ \phi(x) = \frac{1}{\sqrt{2\pi}} e^{-\frac{x^2}{2}} $$

\subparagraph{定理} 设 $ X \sim N(\mu, \sigma^2) $ ,则 $$ Y = \frac{X-\mu}{\sigma} \sim N(0,1) $$


\subsection{随机向量及其分布}

\paragraph{$ n $ 维随机向量} 设 $ (\Omega, \mathcal{F}, P) $ 是一概率空间,
$ \xi_{1}{(\omega)} , \xi_{2}{(\omega)} , \cdots , \xi_{n}{(\omega)}  $ 是定义在这个概率空间上的 $ n $ 个随机变量,
称 $ \xi(\omega) = (\xi_{1}{(\omega)} , \xi_{2}{(\omega)} , \cdots , \xi_{n}{(\omega)}) $ 为 $ n $ 维随机向量。

\paragraph{$ n $ 维随机向量的联合分布} 称 $ n $ 元函数
$$ F_{X_1 , X_2 , \cdots , X_n }(x_{1} , x_{2} , \cdots , x_{n}) = P(\xi_{1} \leqslant x_1 , \xi_{2} \leqslant x_2 , \cdots , \xi_{n} \leqslant x_n ) $$
为随机向量 $ \xi = (\xi_1, \xi_2, \cdots, \xi_n) $ 的联合分布函数。

\subparagraph{二维随机变量分布函数的几何意义}
将二维随机变量 $ (X,Y) $ 看平面上随机点的坐标,$ (X,Y) $ 落在区域 $ \{ x < x_2, y < y_2 \} $ 中的概率为 $ F_{X,Y}(x,y) $

\subparagraph{分布函数的性质} 
\begin{itemize}[leftmargin=\subparitemindent]
    \item 单调不减
    \item $ 0 < F < 1 $
    \item 对于任意 $ x,y $ 有 \begin{align}
        F(x, -\infty) & = \lim_{y \rightarrow -\infty} F(x,y) = 0 \\
        F(-\infty, y) & = \lim_{x \rightarrow -\infty} F(x,y) = 0
    \end{align}
    \item 右连续
\end{itemize}

\subsubsection{离散型随机向量}

\paragraph{离散型随机向量} 若随机向量 $ \xi = (\xi_{1} , \xi_{2} , \cdots , \xi_{n}) $ 只取有限个或可列个不同的向量值,
则称 $ \xi $ 为离散型随机向量。

\paragraph{分布列} 设 $ \xi $ 的所有可能取值为 $ (x_{1i_1} , x_{2i_2} , \cdots , x_{ni_n} ), i_{1} $ ,其中 $ i_{2} , \cdots , i_{n} = 1,2,\cdots $
则称概率
\begin{equation}
    p_{i_{1} , i_{2} , \cdots , i_{n} } = P(\xi_{1} = x_{11_{1}} , \xi_{2} = x_{21_{2}} , \cdots , \xi_{n} = x_{n1_{n}} ) 
\end{equation}
为 $ \xi $ 的分布列。

\subparagraph{分布列的性质} 
\begin{itemize}[leftmargin=\paritemindent]
    \item $ p_{i_{1} , i_{2} , \cdots , i_{n} } \geqslant 0 $
    \item $ \sum_{i_{1} , i_{2} , \cdots , i_{n} } p_{i_{1} , i_{2} , \cdots , i_{n} } = 1 $
\end{itemize}

\subparagraph{分布函数}
\begin{equation}
    F(x_{1} , x_{2} , \cdots , x_{n} ) = \sum_{x_{1i_1} < x_1 , x_{2i_2} < x_2 , \cdots , x_{ni_n} < x_n } p_{i_{1} , i_{2} , \cdots , i_{n} }
\end{equation}

\subsubsection{连续型随机向量}

\paragraph{连续型随机向量} 若存在非负函数 $ p(x_{1} , x_{2} , \cdots , x_{n} ) $ 使得分布函数可表示为
$$ F(x_{1} , x_{2} , \cdots , x_{n} ) = \int_{-\infty}^{x_{1}}  \int_{-\infty}^{x_{2}}  \cdots  \int_{-\infty}^{x_{n}} 
f(s_{1} , s_{2} , \cdots , s_{n}) \diff s_{1} \diff s_{2} \cdots \diff s_{n}  $$ 
则称 $ \xi $ 为连续型随机变量。函数 $ f(s_{1} , s_{2} , \cdots , s_{n}) $ 称为 $ \xi $ 的分布密度或密度函数,满足条件
\begin{itemize}[leftmargin=\paritemindent]
    \item $ f(s_{1} , s_{2} , \cdots , s_{n}) > 0 $
    \item $ \int_{-\infty}^{\infty}  \int_{-\infty}^{\infty}  \cdots  \int_{-\infty}^{\infty} 
    f(s_{1} , s_{2} , \cdots , s_{n}) \diff s_{1} \diff s_{2} \cdots \diff s_{n} = 1$
\end{itemize}
反之,若有满足这两条性质的 $ n $ 元函数 $ f(s_{1} , s_{2} , \cdots , s_{n}) $ 则他一定是某一个 $ n $ 维随机变量的密度函数。

\paragraph{概率} 对于 $ R^n $ 上某一区域 $ B $ 有
\begin{equation}
    P\{ (\xi_{1} , \xi_{2} , \cdots , \xi_{n}) \in B \} = 
    \iint\cdots\int _B f(s_{1} , s_{2} , \cdots , s_{n}) \diff x_{1} , \diff x_{2} , \cdots , \diff x_{n} 
\end{equation}

\paragraph{多元正态分布} 设 $ \symbf{B} = (b_{ij}) $ 是 $ n $ 阶正定对称矩阵,$ \symbf{B}^{-1} = (r_{ij}) $ 是他的逆矩阵,
$ |\symbf{B}| $ 表示 $ \symbf{B} $ 的行列式, $ \symbf{a} = (a_{1} , a_{2} , \cdots , a_{n})^{\symrm{T}} $ 是一个实值列向量,
以函数
\begin{equation}
    f(x_{1}, x_{2}, \cdots, x_{n} ) = \frac{1}{(2\pi)^{\frac{n}{2}} |\symbf{B}|^{\frac{1}{2}}}
    \exp \left\{ -\frac{1}{2} \sum_{i,j=1}^n r_{ij} (x_i - \symbf{a}_i)(x_j - \symbf{a})_j \right\}
\end{equation}
为密度函数的概率分布,称为$ n $元正态分布,简记为 $ N(\symbf{a},\symbf{B}) $。向量形式为
\begin{equation}
    f(x_{1}, x_{2}, \cdots, x_{n} ) = \frac{1}{(2\pi)^{\frac{n}{2}} |\symbf{B}|^{\frac{1}{2}}}
    \exp \left\{ -\frac{1}{2} (\symbf{x} - \symbf{a})^{\symrm{T}} B^{-1} (\symbf{x} - \symbf{a}) \right\}
\end{equation}
其中, $ \symbf{x} = (x_{1}, x_{2}, \cdots, x_{n})^{\symrm{T}} $。

\subparagraph{二元正态分布} 当 $ n=2 $ 时,$ (X,Y) $ 的概率密度为
\begin{equation}
    f(x,y) = \frac{1}{2\pi \sigma_1 \sigma_2 \sqrt{1-\rho^2}}
    \exp \left\{ -\frac{1}{2(1-\rho^2)} \cdot
    \left[ \frac{(x-\mu_1)^2}{\sigma_1^2} -2\rho \frac{(x-\mu_1)(y-\mu_2)}{\sigma_1\sigma_2} + \frac{(y-\mu_2)^2}{\sigma_2^2} \right]
    \right\}
\end{equation}
其中,$ \mu_1, \mu_2, \sigma_1, \sigma_2, \rho $ 都是常数,且 $ \sigma_1 > 0, \sigma_2 > 0, -1 < \rho < 1 $,
则称 $ (X,Y) $ 服从参数为 $ \mu_1, \mu_2, \sigma_1, \sigma_2, \rho $ 的二维正态分布,记为
$$ (X,Y) \sim N(\mu_1, \mu_2, \sigma_1, \sigma_2, \rho) $$

\subsection{边缘分布}

\paragraph{边缘分布} 设 $ (X,Y) $ 是二维随机变量,$ X $ 的分布函数 $ F_X(x) $ 称为 $ (X,Y) $ 关于 $ X $ 的边缘分布函数。
$$ F_X(x) = P\{ X \leqslant x \} = P\{ X \leqslant x, Y < \infty \} = F(x,\infty) $$
同理 $$ F_Y(y) = F(\infty,y) $$

\subsection{条件分布}