\section{假设检验}

\hparagraph{参数假设检验} 总体的分布类型已知,对未知参数作 出假设,用总体中的样本检验此项假设是否 成立,就称为参数假设检验。

\hparagraph{非参数假设检验} 对总体分布函数的形式作出假设,用 总体中的样本检验此项假设是否成立,就称 为非参数假设检验。

\hparagraph{基本步骤}
\begin{itemize}[leftmargin=\paritemindent]
    \item 题出原假设 $ H_0 $ ,确定备择假设 $ H_1 $ ;
    \item 构造分布已知的合适的统计量;
    \item 由给定的检验水平 $ \alpha $ ,求出在 $ H_0 $ 成立的条件下的临界值;
    \item 计算统计量的样本观测值,如果落在拒绝域内,则拒绝原假设;否则,接收原假设。
\end{itemize}

\hparagraph{假设检验的两类错误} 
\hsubparagraph{第一类错误(弃真错误)} 原假设为真,而检验结果拒绝原假设。记为 $ \alpha $ 。
\hsubparagraph{第二类错误(取伪错误)} 原假设为假,而检验结果接受原假设。记为 $ \beta $ 。
\hsubparagraph{原则} 
\begin{itemize}[leftmargin=\subparitemindent]
    \item 在限制 $ \alpha $ 的前提下,使 $ \beta $ 尽可能小。
    \item 往往把不轻易否定的命题作为原假设
    \item 通常控制犯第一类错误的概率
\end{itemize}

\hparagraph{显著性检验} 只对犯第一类错误的概率加以控制,而不考虑犯第二类错误的概率。即求
$$ P\left\{ \mbox{拒绝}H_0 \left| H_0 \mbox{为真} \right. \right\} \leqslant \alpha $$
称 $ \alpha $ 为显著性水平。

\section{正态总体参数的假设检验}

\LTXtable{\linewidth}{Chapter-4/table-4-1.tex}