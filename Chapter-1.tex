\chapter{概率论复习}

\section{概率空间}

\paragraph{样本空间与事件域}
设 $\Omega$ 是样本空间, $\mathcal{F}$ 是由 $\Omega$ 的一些子集构成的集类,如果满足下列条件
\begin{itemize}[leftmargin=\paritemindent]
    \item $\Omega \in \mathcal{F}$
    \item 若 $A\in\mathcal{F}$,则 $\bar{A}\in\mathcal{F}$
    \item 若 $A_1,A_2\ \,\cdots,A_n,\cdots\in\mathcal{F}$,则
     $$ \bigcup_{n=1}^\infty A_n\ \in\mathcal{F} $$ 
\end{itemize}
则 $\mathcal{F}$ 是一个事件域,每个时间发生的概率大小为 $P$,三元体$(\Omega,\mathcal{F},P)$就构成一个概率空间。

\paragraph{概率的性质}
\begin{itemize}[leftmargin=\paritemindent]
    \item $P(\phi)=0$
	\item 可列可加性:如果$A_i\in\mathcal{F}(i=1,2,\cdots,n)$且$A_i\ A_j=\phi(i\neq\ j)$,则
     $$ P\left(\bigcup_{i=1}^n A_i \right)=\sum_{i=1}^n P(A_i )  $$ 
    \item 设$A\in\mathcal{F}$,则有$P(A)=1-P(\bar{A})$
    \item 设$A\in\mathcal{F},B\in\mathcal{F}$,如果$A\supset B$,则$P(A-B)=P(A)-P(B),P(A)\geq P(B)$
    \item 设$A\in\mathcal{F},B\in\mathcal{F}$,则
     $$ P(A\cup B)=P(A)+P(B)-P(AB)\le\ P(A)+P(B) $$ 
    可以推广到$n$个事件的情况
\end{itemize}

\paragraph{条件概率} 设$A,B$是两个随机事件,且$P(A)>0$,则称
\begin{equation}
    P(B | A)=\frac{P(AB)}{P(A)}
\end{equation}
为在事件$A$发生的条件下,事件$B$发生的概率。

\begin{itemize}
    \item 对于任意一个事件 $B$,$P(B|A) \geqslant 0$
    \item $P(\Omega | A)=1$
    \item 设$B_1,B_2,\cdots$互不相容,则 \begin{equation}
        P\left(\left. \bigcup_{i=1}^\infty B_i \right| A \right) = \sum_{i=1}^\infty P(B_i,A)
    \end{equation}
    \item $P(\bar{B} |A) = 1 - P(B|A)$
\end{itemize}

\subparagraph{乘法公式} 设 $A,B \supset \Omega$,当 $P(A) > 0$ 由
\begin{equation*}
    P(B|A) = \frac{P(AB)}{P(A)}
\end{equation*} 得到 \begin{equation}
    P(AB) = P(B|A)P(A) = P(A|B)P(B)
\end{equation} 推广得到
\begin{equation}
    P(A_1,A_2, \cdots ,A_n) = P(A_1)P(A_2|A_1)P(A_3|A_2A_1) \cdots
    P(A_{n}|A_1A_2\cdots A_{n-1})
\end{equation}

\subparagraph{全概率公式} 设随机试验 $E$ 的样本空间为 $\Omega$, $A$ 为 $E$ 的任意一事件,
$B_1,B_2,\cdots,B_n$ 为 $\Omega$ 的一个划分,且 $P(B_i) > 0$, 则
\begin{equation}
    P(A) = P(A|B_1)P(B_1) + P(A|B_2)P(B_2) + \cdots + P(A|B_n)P(B_n)
\end{equation}
在较复杂情况下直接计算$P(A)$不易,但 $A$总是伴随着某些$B_i$ 出现,适当地去构造这一组 $B_i$往往可以使问题简化。

\subparagraph{贝叶斯公式} 设随机试验 $E$ 的样本空间为 $\Omega$,$A \subset \Omega$,
$B_1,B_2,\cdots,B_n$为$\Omega$的一个划分,$P(A) > 0,P(B_i)>0$,则
\begin{equation}
    P(B_i|A) = \frac{P(AB_i)}{P(A)} = \ddfrac{
        P(A|B_i)P(B_i)
    }{
        \sum_{i=1}^n P(A|B_i)P(B_i)
    }
\end{equation}
它是在观察到事件$A$已发生的条件下,寻找导致$A$发 生的每个原因的概率。

\paragraph{事件的独立性}
\subparagraph{两个事件}设 $A,B$ 是两个事件,如果如下等式成立 
\begin{equation}
    P(AB) = P(A)P(B)
\end{equation}
则称事件 $A,B$ 相互独立。

\subparagraph{三个事件} 对于 $A,B,C$ 三个事件,如果如下等式成立
\begin{align}
    \begin{split}
        P(AB) & = P(A)P(B) \\
        P(AC) & = P(A)P(C) \\
        P(BC) & = P(B)P(C) 
    \end{split}
\end{align}
则称 $A,B,C$ 两两独立;如果满足
\begin{equation}
    P(ABC) = P(A)P(B)P(C)
\end{equation}
则称 $A,B,C$ 互相独立。

\subparagraph{多个事件} 设 $A_1,A_2,\cdots,A_n$ 是 $n$ 个事件,若对任意的 $k(2 \leqslant k \leqslant n)$
和任意一组 $ 1 \leqslant i_1 < i_2 < \cdots < i_k \leqslant n $ 都有
\begin{equation}
    P(A_{i_1}, A_{i_2}, \cdots, A_{i_k}) = P(A_{i_1})P(A_{i_2})\cdots P(A_{i_k})
\end{equation}
成立,则称 $n$ 个事件 $A_1,A_2,\cdots,A_n$ 相互独立.

\subparagraph{可数无穷多个} 对于事件序列 $A_1,A_2,\cdots,A_n,\cdots$ 若他们之间任意有限个事件独立,
则称事件序列 $A_1$, $A_2$, $\cdots$, $A_n$, $\cdots$ 独立。

\subparagraph{事件独立的性质} 若 $A_1,A_2,\cdots,A_n$ 独立,则
\begin{itemize}[leftmargin=\subparitemindent]
    \item $A_1',A_2',\cdots,A_n'$ 独立,其中 $A_k' = A_k \wedge \bar{A}_k $
    \item 将事件 $A_1,A_2,\cdots,A_n$ 分成 $k$ 组,
    设 $B_1,B_2,\cdots,B_n$ 分别由第 $1,2,\cdots,k$ 组内的 $A_i$ 经过并、积、差、求余等运算所得,
    则 $B_1,B_2,\cdots,B_n$ 独立。
\end{itemize}

\section{随机变(向)量及其分布}

\subsection{随机变量及其分布}

\paragraph{随机变量} 设随机试验的样本空间  $ \Omega = {\omega} $, $ \xi = \xi(\omega) $
是定义在样本空间 $ \Omega $ 上的\textbf{实值单值函数},称 $ \xi = \xi(\omega) $ 是随机变量。

\subparagraph{随机变量和普通函数的区别}
\begin{itemize}[leftmargin=\subparitemindent]
    \item 定义域不同:随机变量定义在样本空间 $ \Omega $ 上,定义域可以是数也可以不是数;
    而普通函数是定义在实数域上。
    \item 随机变量函数的取值在试验之前无法确定,且取值有一定的概率;而普通函数却没有。
\end{itemize}

\subparagraph{随机变量的分类}
\begin{itemize}[leftmargin=\subparitemindent]
    \item 离散型随机变量
    \item 连续型随机变量
    \item 混合型随机变量
\end{itemize}

\paragraph{分布函数} 设 $ X $ 是一个随机变量, $ x $ 是任意实数,称函数 \begin{equation}
    \label{equ:分布函数定义}
    F(x) = P\{X < x\} \quad (-\infty < x < +\infty)
\end{equation}
为 $ X $ 的分布函数。分布函数 $ F(x) $ 的值就表示 $ X $ 落在区间 $ (-\infty,x] $ 上的概率。

\subparagraph{特点} 
\begin{itemize}[leftmargin=\subparitemindent]
    \item 分布函数完整描述了随机变量的统计规律性
    \item 分布函数是一个普通实值函数 
\end{itemize}

\subparagraph{性质} 以下三条性质是判断函数是否是分布函数的充要条件。
\begin{itemize}[leftmargin=\subparitemindent]
    \item 单调不减
    \item  $ 0 \leqslant F(x) \leqslant 1 $ 且 \begin{align}
        \begin{split}
            F(-\infty) &= \lim_{x \rightarrow -\infty} F(x) = 0 \\
            F(+\infty) &= \lim_{x \rightarrow +\infty} F(x) = 1
        \end{split}
    \end{align}
    \item 右(左)连续性:\begin{align}
        \begin{split}
            F(x - 0) &= \lim_{x \rightarrow x - 0} F(y) = F(x) \\
            F(x + 0) &= \lim_{x \rightarrow x + 0} F(y) = F(x)
        \end{split}
    \end{align}
\end{itemize}

\subparagraph{常用的概率公式}
\begin{itemize}[leftmargin=\subparitemindent]
    \item  $ P(a < X \leqslant b) = P(X \leqslant b) - P(X \leqslant a) = F(b) - F(a) $ 
    \item  $ P(X = x_0) = P(X \leqslant x_0) - P(X < x_0) = F(x_0) - F(x_0 - 0) $ 
    \item  $ P(X \geqslant x_0) = 1 - P(X < x_0) = 1 - F(x_0 - 0) $ 
    \item  $ P(X > x_0) = 1 - P(X \leqslant x_0) = 1 - F(x_0) $ 
\end{itemize}

\subsubsection{离散型随机变量及其分布}

\paragraph{离散型随机变量} 若随机变量 $ X $ 的全部可能取值是有限个或可列无限多个,
则称此随机变量是离散型随机变量。

\paragraph{分布律} 设离散型随机变量 $ X $ 的所有可能取值为 $ x_k,k=(1,2,\cdots) $ ,
 $ X $ 取各个可能值的概率为 $ P(X = x_k) = p_k $ , $ p_k $ 满足
\begin{itemize}[leftmargin=\paritemindent]
    \item  $ p_k \geqslant 0 $ 
    \item  $ \sum_{k=1}^\infty p_k = 1 $ 
\end{itemize}
则称 $ p_k $ 为离散型随机变量X的概率分布或分布律。

\paragraph{常用的离散型随机变量分布}

\subparagraph{0—1分布}  $ P(X = k) = p^k(1-p)^{1-k}, k = 0,1 $ 

\subparagraph{二项分布}  $ P(X=k) = C_n^k p^k (1-p)^{n-k}, k = 0,1,\cdots,n. $ 记为 $ X \sim b(n,p) $
\begin{itemize}[leftmargin=\subparitemindent]
    \item \textbf{含义}:   $ n $  重贝努里试验中出现“成功”次数 $ X $ 的概率分布。
    \item \textbf{退化为0—1分布}:  $ n=1 $ 
\end{itemize}

\subparagraph{泊松分布}  分布律  $$ P(X=k) = \frac{\lambda^k}{k!}e^{-\lambda} \quad k = 0,1,2,\cdots $$ 
其中  $ \lambda $ 是常数。记为  $ X \sim P(\lambda) $ 。
泊松分布在管理科学、运筹学以及自然科学的某些问题中都占有重要的地位。 
\begin{itemize}[leftmargin=\subparitemindent]
    \item 排队问题:在一段时间内窗口等待服务的顾客人数
    \item 生物存活的个数
    \item 放射的粒子数
\end{itemize}

\subsubsection{连续型随机变量及其分布}

\paragraph{连续性随机变量} 如果随机变量 $ X $ 的分布函数为
$$ F(x) = \int_{-\infty}^x f(t)\diff t $$ 
其中被积函数 $ f(t) \geqslant 0 $ 则 $ X $ 为连续型随机变量,
称 $ f(t) $ 为概率密度函数或概率密度。

\paragraph{概率密度的性质}
\begin{itemize}[leftmargin=\paritemindent]
    \item  $ f(x) \geqslant 0 $ 
    \item  $ \int_{-\infty}^\infty f(x) \diff x = 1 $ 
    \item  $ P(a < X \leqslant b) = F(b) - F(a) = \int_a^b f(x) \diff x $
    \item  $ P(X = a) = 0 $ 
    \item 在 $ f(x) $ 的连续点 $ x $ 处,有\begin{align}
        \begin{split}
            f(x) = F'(x)
            & = \lim_{\Delta x \rightarrow 0^+} \frac{F(x + \Delta x) - F(x)}{\Delta x} \\
            & = \lim_{\Delta x \rightarrow 0^+} \frac{P(x < X \leqslant (x + \Delta x))}{\Delta x} \\
        \end{split}
    \end{align}
    \item  $ P(a \leqslant X \leqslant b) = P(a < X \leqslant b) = P(a \leqslant X < b) = P(a < X < b) $ 
    \item 若已知连续型随机变量 $ X $ 的密度函数为 $ f(x) $ ,则 $ X $ 在任意区间 $ G $ 上取值的概率为
    \begin{equation}
        P{X \in G} = \int_G f(x) \diff x
    \end{equation}
\end{itemize}

\paragraph{常见的连续型随机变量的分布}

\subparagraph{均匀分布}  $ X $ 的概率密度为:
\begin{equation}
    f(x) = \left\{ \begin{array}{ll}
        \ddfrac{1}{b-a}, & x \in (a,b) \\
        0, & x \in (-\infty, a] \cup [b, \infty)
    \end{array} \right.
\end{equation}
则称 $ X $ 服从 $ (a,b) $ 上的均匀分布,记为 $ X \sim U(a,b) $ 。分布函数为
\begin{equation}
    F(x) = \int_{-\infty}^x f(t) \diff t = \left\{ \begin{array}{ll}
        0, & x < a \\
        \ddfrac{x-a}{b-a}, & a \leqslant x < b  \\
        1, & x \geqslant b
    \end{array} \right.
\end{equation}

\subsection{随机向量及其分布}

\subsection{边缘分布}

\subsection{条件分布}