\section{回归分析的基本概念}

\hparagraph{函数关系与相关关系}

\hsubparagraph{函数关系} 变量之间确实存在的,且在数量上表现为确定性的相互依存关系。

\hsubparagraph{相关关系} 变量之间确实存在的,但在数量上表现为不确定的相互依存关系。
\begin{equation}
    y = f(x_1, x_2, \cdots, x_k) + \epsilon
\end{equation}
其中, $ x_1, x_2, \cdots, x_k $ 称为解释变量(自变量), $ y $ 称为被解释变量(响应变量、因变量),
$ \epsilon $ 是其他随即影响因素,通常假设是不可观测的随机误差,是一个随机变量。

\hparagraph{回归分析} 回归分析就是研究相关关系的一种重要的数理统计方法。

\hsubparagraph{分类} 

\begin{itemize}[leftmargin=\subparitemindent]
    \item 当自变量只有两个时,称为一元回归分析;当自变量在两个以上时,称为多元回归分析。
    \item 变量间成线性关系, 称线性回归;变量间不具有线性关系,称非线性回归。
\end{itemize}

\hsubparagraph{多元线性回归模型} \begin{equation}
    y = \beta_0 + \beta_1 x_1, \beta_2 x_2, \cdots, \beta_k x_k + \epsilon
\end{equation}
通常假定 $ \epsilon \sim N(0, \sigma^2) $ 。

